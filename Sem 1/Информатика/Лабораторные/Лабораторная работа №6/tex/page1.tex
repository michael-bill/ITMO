\twocolumn
\textbf{\large{Математический кружок}} \\
\rule{\textwidth}{2pt}
\includegraphics[width=2cm, height=2cm]{image.png} \\\\
\textit{\large{Ю. Ионин, Л. Курляндчик}} \\
\Huge{\textbf{Поиск}} \\
\Huge{\textbf{инварианта}} \\
\parindent = 0pt
\par\large\textbf{В этой статье рассматриваются задачи, сходные по формулировке, В каждой из них идет речь о некоторой совокупности чисел или знаков, н указаны операции, которые можно над ними производить. Читатель, внимательно проработавший статью, обнаружит не только внешнее сходство встречающихся в ней задач и упражнений, но и общую идею, на которой основано них решение.}
\\
\par\Large{З а д а ч а  1. \textit{На доске написано десять плюсов и пятнадцать минусов. Разрешается стереть любые два знака и написать вместо них плюс, если они одинаковы, и минус в противном случае. Какой знак останется на доске после выполнения двадцати четырех таких операций?}}
\parindent = 20pt
\par\Large{Р е ш е н и е. Заменим каждый плюс числом 1, а каждый минус числом - 1. Разрешенная операция описывается тогда так: стираются любые два числа и записывается их произведение. Поэтмоу произведение всех написанных на доске чисел остается неизменным. Так как вначале это проиведение равнялось -1, то и в конце останется число -1, то есть знак минус.}
\par\Large{Это рассуждение можно было провести иначе. Заменим все плюсы ну-\\\\\\\\лями, а минусы - единицами, и заметим, что сумма двух стираемых чисел имеет ту же четность, что и число, записываемое вместо них. Так как сначала сумма всех чисел была начетной (она равнялась 15), то и последнее оставшееся на доске число будет нечетным, то есть единицей, и, значит, на доске останется минус.}
\par\Large{Наконец, третье решение задачи можно получить заметив, что в результате каждой операции число минусов либо не изменяется, либо уменьшается на два. Поскольку сначала число минусов было нечетным, то и в конце останется один минус.}
\par\Large{Проанализируем все три решения.}
\par\Large{Первое решение основывалось на неизменяемости произведения написанных чисел, второе - на неизменяемости четности их суммы и третье - на неизменяемости четности числа минусов. В математике вместо слова «\textit{неизменяемость}» употребляют термин «\textit{инвариантность}». Можно сказать, что в каждом решении нам удалось найти \textit{инвариант}: проиведение написанных чисел, четность суммы, четность числа минусов. Решение последующих задач и упражнений также основывается на удачном подборе инварианта.}
\par\large{У п р а ж н е н и е  1. На доске написано несколько плюсов и минусов. Разрешается стереть любые два знака и написать вместо них плюс, если они различны, и минус в противном случае. Докажите, что последний оставшийся на доске знак не зависит от порядке, в котором производились стирания.}
\par\Large{З а д а ч а  2.   В \textit{таблице} $4\times{}4$ \textit{знаки} «$+$» и «$-$» \textit{расставлены так, как показано на рисунке} 1.}