\documentclass{article}

\usepackage[a4paper,left=2cm,right=2cm,top=2cm,bottom=1cm,footskip=.5cm]{geometry}

\usepackage{fontspec}
\setmainfont{CMU Serif}
\setsansfont{CMU Sans Serif}
\setmonofont{CMU Typewriter Text}

\usepackage[russian]{babel}

\usepackage{mathtools}
\usepackage{karnaugh-map}
\usepackage{tikz}
\usetikzlibrary {circuits.logic.IEC}

\begin{document}

\begin{center}
    УНИВЕРСИТЕТ ИТМО \\
    Факультет программной инженерии и компьютерной техники \\
    Дисциплина «Дискретная математика»
    
    \vspace{5cm}

    \large
    \textbf{Курсовая работа} \\
    Часть 2 \\
    Вариант 95
\end{center}

\vspace{2cm}

\hfill\begin{minipage}{0.35\linewidth}
Студент \\
XXX XXX XXX \\
Р31XX \\

Преподаватель \\
Поляков Владимир Иванович
\end{minipage}

\vfill

\begin{center}
    Санкт-Петербург, 2023 г.
\end{center}

\thispagestyle{empty}
\newpage

\section*{Задание}
Построить комбинационную схему реализующую функцию $C = A * B$ ($C$ --- 5 бит, $A$ --- 3 бита, $B$ --- 2 бит) $A \neq 0$ и $B \neq 0$.
\section*{Таблица истинности}
\begin{center}\begin{tabular}{|c|ccc|cc|ccccc|}
    \hline № & $a_1$ & $a_2$ & $a_3$ & $b_1$ & $b_2$ & $c_1$ & $c_2$ & $c_3$ & $c_4$ & $c_5$ \\ \hline
    0 & 0 & 0 & 0 & 0 & 0 & d & d & d & d & d \\ \hline
    1 & 0 & 0 & 0 & 0 & 1 & d & d & d & d & d \\ \hline
    2 & 0 & 0 & 0 & 1 & 0 & d & d & d & d & d \\ \hline
    3 & 0 & 0 & 0 & 1 & 1 & d & d & d & d & d \\ \hline
    4 & 0 & 0 & 1 & 0 & 0 & d & d & d & d & d \\ \hline
    5 & 0 & 0 & 1 & 0 & 1 & 0 & 0 & 0 & 0 & 1 \\ \hline
    6 & 0 & 0 & 1 & 1 & 0 & 0 & 0 & 0 & 1 & 0 \\ \hline
    7 & 0 & 0 & 1 & 1 & 1 & 0 & 0 & 0 & 1 & 1 \\ \hline
    8 & 0 & 1 & 0 & 0 & 0 & d & d & d & d & d \\ \hline
    9 & 0 & 1 & 0 & 0 & 1 & 0 & 0 & 0 & 1 & 0 \\ \hline
    10 & 0 & 1 & 0 & 1 & 0 & 0 & 0 & 1 & 0 & 0 \\ \hline
    11 & 0 & 1 & 0 & 1 & 1 & 0 & 0 & 1 & 1 & 0 \\ \hline
    12 & 0 & 1 & 1 & 0 & 0 & d & d & d & d & d \\ \hline
    13 & 0 & 1 & 1 & 0 & 1 & 0 & 0 & 0 & 1 & 1 \\ \hline
    14 & 0 & 1 & 1 & 1 & 0 & 0 & 0 & 1 & 1 & 0 \\ \hline
    15 & 0 & 1 & 1 & 1 & 1 & 0 & 1 & 0 & 0 & 1 \\ \hline
    16 & 1 & 0 & 0 & 0 & 0 & d & d & d & d & d \\ \hline
    17 & 1 & 0 & 0 & 0 & 1 & 0 & 0 & 1 & 0 & 0 \\ \hline
    18 & 1 & 0 & 0 & 1 & 0 & 0 & 1 & 0 & 0 & 0 \\ \hline
    19 & 1 & 0 & 0 & 1 & 1 & 0 & 1 & 1 & 0 & 0 \\ \hline
    20 & 1 & 0 & 1 & 0 & 0 & d & d & d & d & d \\ \hline
    21 & 1 & 0 & 1 & 0 & 1 & 0 & 0 & 1 & 0 & 1 \\ \hline
    22 & 1 & 0 & 1 & 1 & 0 & 0 & 1 & 0 & 1 & 0 \\ \hline
    23 & 1 & 0 & 1 & 1 & 1 & 0 & 1 & 1 & 1 & 1 \\ \hline
    24 & 1 & 1 & 0 & 0 & 0 & d & d & d & d & d \\ \hline
    25 & 1 & 1 & 0 & 0 & 1 & 0 & 0 & 1 & 1 & 0 \\ \hline
    26 & 1 & 1 & 0 & 1 & 0 & 0 & 1 & 1 & 0 & 0 \\ \hline
    27 & 1 & 1 & 0 & 1 & 1 & 1 & 0 & 0 & 1 & 0 \\ \hline
    28 & 1 & 1 & 1 & 0 & 0 & d & d & d & d & d \\ \hline
    29 & 1 & 1 & 1 & 0 & 1 & 0 & 0 & 1 & 1 & 1 \\ \hline
    30 & 1 & 1 & 1 & 1 & 0 & 0 & 1 & 1 & 1 & 0 \\ \hline
    31 & 1 & 1 & 1 & 1 & 1 & 1 & 0 & 1 & 0 & 1 \\ \hline
\end{tabular}\end{center}

\section*{Минимизация булевых функций на картах Карно}
\noindent\begin{minipage}{\textwidth}
\begin{karnaugh-map}[4][4][2][$b_1$$b_2$][$a_2$$a_3$][$a_1$]
    \minterms{27,31}
    \terms{0,1,2,3,4,8,12,16,20,24,28}{d}
    \implicant{15}{11}[1]
\end{karnaugh-map}
\[c_1 = a_1\,a_2\,b_1\,b_2 \quad (S_Q = 4)\] \\ \phantom{0}
\end{minipage}
\noindent\begin{minipage}{\textwidth}
\begin{karnaugh-map}[4][4][2][$b_1$$b_2$][$a_2$$a_3$][$a_1$]
    \maxterms{5,6,7,9,10,11,13,14,17,21,25,27,29,31}
    \terms{0,1,2,3,4,8,12,16,20,24,28}{d}
    \implicant{0}{9}[0,1]
    \implicant{0}{6}[0]
    \implicantedge{0}{2}{8}{10}[0]
    \implicantedge{0}{8}{2}{10}[0]
    \implicant{13}{11}[1]
\end{karnaugh-map}
\[c_2 = b_1\,\left(a_1 \lor a_2\right)\,\left(a_1 \lor a_3\right)\,\left(a_1 \lor b_2\right)\,\left(\overline{a_1} \lor \overline{a_2} \lor \overline{b_2}\right) \quad (S_Q = 14)\] \\ \phantom{0}
\end{minipage}
\noindent\begin{minipage}{\textwidth}
\begin{karnaugh-map}[4][4][2][$b_1$$b_2$][$a_2$$a_3$][$a_1$]
    \maxterms{5,6,7,9,13,15,18,22,27}
    \terms{0,1,2,3,4,8,12,16,20,24,28}{d}
    \implicant{0}{9}[0]
    \implicantedge{0}{4}{2}{6}[0,1]
    \implicant{5}{15}[0]
    \implicant{11}{11}[1]
\end{karnaugh-map}
\[c_3 = \left(a_1 \lor b_1\right)\,\left(a_2 \lor b_2\right)\,\left(a_1 \lor \overline{a_3} \lor \overline{b_2}\right)\,\left(\overline{a_1} \lor \overline{a_2} \lor a_3 \lor \overline{b_1} \lor \overline{b_2}\right) \quad (S_Q = 16)\] \\ \phantom{0}
\end{minipage}
\noindent\begin{minipage}{\textwidth}
\begin{karnaugh-map}[4][4][2][$b_1$$b_2$][$a_2$$a_3$][$a_1$]
    \maxterms{5,10,15,17,18,19,21,26,31}
    \terms{0,1,2,3,4,8,12,16,20,24,28}{d}
    \implicant{0}{2}[0,1]
    \implicant{0}{5}[0,1]
    \implicantcorner[0,1]
    \implicant{15}{15}[0,1]
\end{karnaugh-map}
\[c_4 = \left(a_2 \lor a_3\right)\,\left(a_2 \lor b_1\right)\,\left(a_3 \lor b_2\right)\,\left(\overline{a_2} \lor \overline{a_3} \lor \overline{b_1} \lor \overline{b_2}\right) \quad (S_Q = 14)\] \\ \phantom{0}
\end{minipage}
\noindent\begin{minipage}{\textwidth}
\begin{karnaugh-map}[4][4][2][$b_1$$b_2$][$a_2$$a_3$][$a_1$]
    \minterms{5,7,13,15,21,23,29,31}
    \terms{0,1,2,3,4,8,12,16,20,24,28}{d}
    \implicant{5}{15}[0,1]
\end{karnaugh-map}
\[c_5 = a_3\,b_2 \quad (S_Q = 2)\] \\ \phantom{0}
\end{minipage}
\section*{Преобразование системы булевых функций}
\[\begin{matrix}
    \begin{cases}
        c_1 = a_1\,a_2\,b_1\,b_2 & (S_Q^{c_1} = 4) \\
        c_2 = b_1\,\left(a_1 \lor a_2\right)\,\left(a_1 \lor a_3\right)\,\left(a_1 \lor b_2\right)\,\left(\overline{a_1} \lor \overline{a_2} \lor \overline{b_2}\right) & (S_Q^{c_2} = 14) \\
        c_3 = \left(a_1 \lor b_1\right)\,\left(a_2 \lor b_2\right)\,\left(a_1 \lor \overline{a_3} \lor \overline{b_2}\right)\,\left(\overline{a_1} \lor \overline{a_2} \lor a_3 \lor \overline{b_1} \lor \overline{b_2}\right) & (S_Q^{c_3} = 16) \\
        c_4 = \left(a_2 \lor a_3\right)\,\left(a_2 \lor b_1\right)\,\left(a_3 \lor b_2\right)\,\left(\overline{a_2} \lor \overline{a_3} \lor \overline{b_1} \lor \overline{b_2}\right) & (S_Q^{c_4} = 14) \\
        c_5 = a_3\,b_2 & (S_Q^{c_5} = 2) \\
    \end{cases} \\ (S_Q = 50)
\end{matrix}\] \\ \phantom{0}
\noindent\begin{minipage}{\textwidth}
Проведем раздельную факторизацию системы.
\[\begin{matrix}
    \begin{cases}
        c_1 = a_1\,a_2\,b_1\,b_2 & (S_Q^{c_1} = 4) \\
        c_2 = b_1\,\left(a_1 \lor a_2\,a_3\,b_2\right)\,\left(\overline{a_1} \lor \overline{a_2} \lor \overline{b_2}\right) & (S_Q^{c_2} = 11) \\
        c_3 = \left(a_2 \lor b_2\right)\,\left(a_1 \lor b_1\,\left(\overline{a_3} \lor \overline{b_2}\right)\right)\,\left(\overline{a_1} \lor \overline{a_2} \lor a_3 \lor \overline{b_1} \lor \overline{b_2}\right) & (S_Q^{c_3} = 16) \\
        c_4 = \left(a_2 \lor b_1\right)\,\left(a_3 \lor a_2\,b_2\right)\,\left(\overline{a_2} \lor \overline{a_3} \lor \overline{b_1} \lor \overline{b_2}\right) & (S_Q^{c_4} = 13) \\
        c_5 = a_3\,b_2 & (S_Q^{c_5} = 2) \\
    \end{cases} \\ (S_Q = 46)
\end{matrix}\] \\ \phantom{0}
\end{minipage}
\noindent\begin{minipage}{\textwidth}
Проведем совместную декомпозицию системы. \[\varphi_{0} = a_1\,a_2\,b_2, \quad \overline{\varphi_{0}} = \overline{a_1} \lor \overline{a_2} \lor \overline{b_2}\]
\[\begin{matrix}
    \begin{cases}
        \varphi_{0} = a_1\,a_2\,b_2 & (S_Q^{\varphi_{0}} = 3) \\
        c_1 = \varphi_{0}\,b_1 & (S_Q^{c_1} = 2) \\
        c_2 = \overline{\varphi_{0}}\,b_1\,\left(a_1 \lor a_2\,a_3\,b_2\right) & (S_Q^{c_2} = 8) \\
        c_3 = \left(a_1 \lor b_1\,\left(\overline{a_3} \lor \overline{b_2}\right)\right)\,\left(a_2 \lor b_2\right)\,\left(\overline{\varphi_{0}} \lor a_3 \lor \overline{b_1}\right) & (S_Q^{c_3} = 14) \\
        c_4 = \left(a_2 \lor b_1\right)\,\left(a_3 \lor a_2\,b_2\right)\,\left(\overline{a_2} \lor \overline{a_3} \lor \overline{b_1} \lor \overline{b_2}\right) & (S_Q^{c_4} = 13) \\
        c_5 = a_3\,b_2 & (S_Q^{c_5} = 2) \\
    \end{cases} \\ (S_Q = 43)
\end{matrix}\] \\ \phantom{0}
\end{minipage}
\noindent\begin{minipage}{\textwidth}
Проведем совместную декомпозицию системы. \[c_5 = a_3\,b_2, \quad \overline{c_5} = \overline{a_3} \lor \overline{b_2}\]
\[\begin{matrix}
    \begin{cases}
        c_5 = a_3\,b_2 & (S_Q^{c_5} = 2) \\
        \varphi_{0} = a_1\,a_2\,b_2 & (S_Q^{\varphi_{0}} = 3) \\
        c_1 = \varphi_{0}\,b_1 & (S_Q^{c_1} = 2) \\
        c_2 = \overline{\varphi_{0}}\,b_1\,\left(a_1 \lor a_2\,c_5\right) & (S_Q^{c_2} = 7) \\
        c_3 = \left(a_1 \lor b_1\,\overline{c_5}\right)\,\left(a_2 \lor b_2\right)\,\left(\overline{\varphi_{0}} \lor a_3 \lor \overline{b_1}\right) & (S_Q^{c_3} = 12) \\
        c_4 = \left(a_2 \lor b_1\right)\,\left(a_3 \lor a_2\,b_2\right)\,\left(\overline{a_2} \lor \overline{b_1} \lor \overline{c_5}\right) & (S_Q^{c_4} = 12) \\
    \end{cases} \\ (S_Q = 40)
\end{matrix}\] \\ \phantom{0}
\end{minipage}
\noindent\begin{minipage}{\textwidth}
Проведем совместную декомпозицию системы. \[\varphi_{1} = a_2\,b_2\]
\[\begin{matrix}
    \begin{cases}
        \varphi_{1} = a_2\,b_2 & (S_Q^{\varphi_{1}} = 2) \\
        c_5 = a_3\,b_2 & (S_Q^{c_5} = 2) \\
        \varphi_{0} = \varphi_{1}\,a_1 & (S_Q^{\varphi_{0}} = 2) \\
        c_1 = \varphi_{0}\,b_1 & (S_Q^{c_1} = 2) \\
        c_2 = \overline{\varphi_{0}}\,b_1\,\left(a_1 \lor a_2\,c_5\right) & (S_Q^{c_2} = 7) \\
        c_3 = \left(a_1 \lor b_1\,\overline{c_5}\right)\,\left(a_2 \lor b_2\right)\,\left(\overline{\varphi_{0}} \lor a_3 \lor \overline{b_1}\right) & (S_Q^{c_3} = 12) \\
        c_4 = \left(a_2 \lor b_1\right)\,\left(\varphi_{1} \lor a_3\right)\,\left(\overline{a_2} \lor \overline{b_1} \lor \overline{c_5}\right) & (S_Q^{c_4} = 10) \\
    \end{cases} \\ (S_Q = 39)
\end{matrix}\] \\ \phantom{0}
\end{minipage}
\clearpage
\section*{Синтез комбинационной схемы в булемов базисе}
Будем анализировать схему на следующем наборе аргументов:
\[a_1 = 1,\:a_2 = 0,\:a_3 = 1,\:b_1 = 1,\:b_2 = 0\]
Выходы схемы из таблицы истинности:
\[c_1 = \text{0},\:c_2 = \text{1},\:c_3 = \text{0},\:c_4 = \text{1},\:c_5 = \text{0}\]
\begin{center}\begin{tikzpicture}[circuit logic IEC]
\node[and gate,inputs={nn}] at (0,-0.5) (n1) {};
\node at (-1.5,-0.6666667) (n2) {$b_2$};
\draw (n1.input 2) -- ++(left:2mm) |- (n2.east) node[at end, above, xshift=2.0mm, yshift=-2pt]{\scriptsize $0$};
\node at (-1.5,-0.33333334) (n3) {$a_2$};
\draw (n1.input 1) -- ++(left:2mm) |- (n3.east) node[at end, above, xshift=2.0mm, yshift=-2pt]{\scriptsize $0$};
\node[and gate,inputs={nn}] at (0,-2.5) (n4) {};
\node at (-1.5,-2.6666665) (n5) {$b_2$};
\draw (n4.input 2) -- ++(left:2mm) |- (n5.east) node[at end, above, xshift=2.0mm, yshift=-2pt]{\scriptsize $0$};
\node at (-1.5,-2.333333) (n6) {$a_3$};
\draw (n4.input 1) -- ++(left:2mm) |- (n6.east) node[at end, above, xshift=2.0mm, yshift=-2pt]{\scriptsize $1$};
\node[and gate,inputs={nn}] at (0,-4.5) (n7) {};
\node at (-1.5,-4.6666665) (n8) {$a_1$};
\draw (n7.input 2) -- ++(left:2mm) |- (n8.east) node[at end, above, xshift=2.0mm, yshift=-2pt]{\scriptsize $1$};
\node at (-1.5,-4.333333) (n9) {$\varphi_{1}$};
\draw (n7.input 1) -- ++(left:2mm) |- (n9.east) node[at end, above, xshift=2.0mm, yshift=-2pt]{\scriptsize $0$};
\node[and gate,inputs={nn}] at (0,-6.5) (n10) {};
\node at (-1.5,-6.6666665) (n11) {$b_1$};
\draw (n10.input 2) -- ++(left:2mm) |- (n11.east) node[at end, above, xshift=2.0mm, yshift=-2pt]{\scriptsize $1$};
\node at (-1.5,-6.333333) (n12) {$\varphi_{0}$};
\draw (n10.input 1) -- ++(left:2mm) |- (n12.east) node[at end, above, xshift=2.0mm, yshift=-2pt]{\scriptsize $0$};
\node[and gate,inputs={nnn}] at (0,-9) (n13) {};
\node[or gate,inputs={nn}] at (-1.5,-9.333333) (n14) {};
\node[and gate,inputs={nn}] at (-3,-9.5) (n15) {};
\node at (-4.5,-9.666667) (n16) {$c_5$};
\draw (n15.input 2) -- ++(left:2mm) |- (n16.east) node[at end, above, xshift=2.0mm, yshift=-2pt]{\scriptsize $0$};
\node at (-4.5,-9.333334) (n17) {$a_2$};
\draw (n15.input 1) -- ++(left:2mm) |- (n17.east) node[at end, above, xshift=2.0mm, yshift=-2pt]{\scriptsize $0$};
\draw (n14.input 2) -- ++(left:2mm) |- (n15.output) node[at end, above, xshift=2.0mm, yshift=-2pt]{\scriptsize $0$};
\node at (-3,-8.783333) (n18) {$a_1$};
\draw (n14.input 1) -- ++(left:2mm) |- (n18.east) node[at end, above, xshift=2.0mm, yshift=-2pt]{\scriptsize $1$};
\draw (n13.input 3) -- ++(left:2mm) |- (n14.output) node[at end, above, xshift=2.0mm, yshift=-2pt]{\scriptsize $1$};
\node at (-1.5,-8.45) (n19) {$b_1$};
\draw (n13.input 2) -- ++(left:3.5mm) |- (n19.east) node[at end, above, xshift=2.0mm, yshift=-2pt]{\scriptsize $1$};
\node at (-1.5,-8.116667) (n20) {$\overline{\varphi_{0}}$};
\draw (n13.input 1) -- ++(left:2mm) |- (n20.east) node[at end, above, xshift=2.0mm, yshift=-2pt]{\scriptsize $1$};
\node[and gate,inputs={nnn}] at (0,-12.766666) (n21) {};
\node[or gate,inputs={nnn}] at (-1.5,-14.033333) (n22) {};
\node at (-3,-14.366666) (n23) {$\overline{b_1}$};
\draw (n22.input 3) -- ++(left:2mm) |- (n23.east) node[at end, above, xshift=2.0mm, yshift=-2pt]{\scriptsize $0$};
\node at (-3,-14.033333) (n24) {$a_3$};
\draw (n22.input 2) -- ++(left:3.5mm) |- (n24.east) node[at end, above, xshift=2.0mm, yshift=-2pt]{\scriptsize $1$};
\node at (-3,-13.7) (n25) {$\overline{\varphi_{0}}$};
\draw (n22.input 1) -- ++(left:2mm) |- (n25.east) node[at end, above, xshift=2.0mm, yshift=-2pt]{\scriptsize $1$};
\draw (n21.input 3) -- ++(left:2mm) |- (n22.output) node[at end, above, xshift=2.0mm, yshift=-2pt]{\scriptsize $1$};
\node[or gate,inputs={nn}] at (-1.5,-12.933332) (n26) {};
\node at (-3,-13.099999) (n27) {$b_2$};
\draw (n26.input 2) -- ++(left:2mm) |- (n27.east) node[at end, above, xshift=2.0mm, yshift=-2pt]{\scriptsize $0$};
\node at (-3,-12.766666) (n28) {$a_2$};
\draw (n26.input 1) -- ++(left:2mm) |- (n28.east) node[at end, above, xshift=2.0mm, yshift=-2pt]{\scriptsize $0$};
\draw (n21.input 2) -- ++(left:3.5mm) |- (n26.output) node[at end, above, xshift=2.0mm, yshift=-2pt]{\scriptsize $0$};
\node[or gate,inputs={nn}] at (-1.5,-11.666665) (n29) {};
\node[and gate,inputs={nn}] at (-3,-11.833332) (n30) {};
\node at (-4.5,-11.999999) (n31) {$\overline{c_5}$};
\draw (n30.input 2) -- ++(left:2mm) |- (n31.east) node[at end, above, xshift=2.0mm, yshift=-2pt]{\scriptsize $1$};
\node at (-4.5,-11.666666) (n32) {$b_1$};
\draw (n30.input 1) -- ++(left:2mm) |- (n32.east) node[at end, above, xshift=2.0mm, yshift=-2pt]{\scriptsize $1$};
\draw (n29.input 2) -- ++(left:2mm) |- (n30.output) node[at end, above, xshift=2.0mm, yshift=-2pt]{\scriptsize $1$};
\node at (-3,-11.116665) (n33) {$a_1$};
\draw (n29.input 1) -- ++(left:2mm) |- (n33.east) node[at end, above, xshift=2.0mm, yshift=-2pt]{\scriptsize $1$};
\draw (n21.input 1) -- ++(left:2mm) |- (n29.output) node[at end, above, xshift=2.0mm, yshift=-2pt]{\scriptsize $1$};
\node[and gate,inputs={nnn}] at (0,-17.133333) (n34) {};
\node[or gate,inputs={nnn}] at (-1.5,-18.233334) (n35) {};
\node at (-3,-18.566668) (n36) {$\overline{c_5}$};
\draw (n35.input 3) -- ++(left:2mm) |- (n36.east) node[at end, above, xshift=2.0mm, yshift=-2pt]{\scriptsize $1$};
\node at (-3,-18.233334) (n37) {$\overline{b_1}$};
\draw (n35.input 2) -- ++(left:3.5mm) |- (n37.east) node[at end, above, xshift=2.0mm, yshift=-2pt]{\scriptsize $0$};
\node at (-3,-17.9) (n38) {$\overline{a_2}$};
\draw (n35.input 1) -- ++(left:2mm) |- (n38.east) node[at end, above, xshift=2.0mm, yshift=-2pt]{\scriptsize $1$};
\draw (n34.input 3) -- ++(left:2mm) |- (n35.output) node[at end, above, xshift=2.0mm, yshift=-2pt]{\scriptsize $1$};
\node[or gate,inputs={nn}] at (-1.5,-17.133333) (n39) {};
\node at (-3,-17.3) (n40) {$a_3$};
\draw (n39.input 2) -- ++(left:2mm) |- (n40.east) node[at end, above, xshift=2.0mm, yshift=-2pt]{\scriptsize $1$};
\node at (-3,-16.966665) (n41) {$\varphi_{1}$};
\draw (n39.input 1) -- ++(left:2mm) |- (n41.east) node[at end, above, xshift=2.0mm, yshift=-2pt]{\scriptsize $0$};
\draw (n34.input 2) -- ++(left:3.5mm) |- (n39.output) node[at end, above, xshift=2.0mm, yshift=-2pt]{\scriptsize $1$};
\node[or gate,inputs={nn}] at (-1.5,-16.033333) (n42) {};
\node at (-3,-16.199999) (n43) {$b_1$};
\draw (n42.input 2) -- ++(left:2mm) |- (n43.east) node[at end, above, xshift=2.0mm, yshift=-2pt]{\scriptsize $1$};
\node at (-3,-15.866666) (n44) {$a_2$};
\draw (n42.input 1) -- ++(left:2mm) |- (n44.east) node[at end, above, xshift=2.0mm, yshift=-2pt]{\scriptsize $0$};
\draw (n34.input 1) -- ++(left:2mm) |- (n42.output) node[at end, above, xshift=2.0mm, yshift=-2pt]{\scriptsize $1$};
\draw (n1.output) -- ++(right:15mm) node[midway, above, yshift=-2pt]{\scriptsize $\varphi_{1} = 0$};
\draw (1.8125,-0.5) -- (1.8125,-1.25);
\draw (1.8125,-1.25) -- (-6,-1.25);
\node[circle, fill=black, inner sep=0pt, minimum size=3pt] (c0) at (-6,-4.333333) {};
\draw (-6,-4.333333) -- (n9.west);
\draw (-6,-16.966665) -- (n41.west);
\draw (-6,-16.966665) -- (-6,-1.25);
\draw (n4.output) -- ++(right:15mm) node[midway, above, yshift=-2pt]{\scriptsize $c_5 = 0$};
\node[not gate] at (2.125,-2.5) (n45) {};
\draw (n45.output) -- (3.0,-2.5);
\node[circle, fill=black, inner sep=0pt, minimum size=3pt] (c0) at (1.0625,-2.5) {};
\draw (3,-2.5) -- (3,-3.5);
\draw (3,-3.5) -- (-6.25,-3.5);
\node[circle, fill=black, inner sep=0pt, minimum size=3pt] (c0) at (-6.25,-11.999999) {};
\draw (-6.25,-11.999999) -- (n31.west);
\draw (-6.25,-18.566668) -- (n36.west);
\draw (-6.25,-18.566668) -- (-6.25,-3.5);
\draw (1.0625,-2.5) -- (1.0625,-3.25);
\draw (1.0625,-3.25) -- (-6.5,-3.25);
\node[circle, fill=black, inner sep=0pt, minimum size=3pt] (c0) at (1.0625,-3.25) {};
\draw (1.0625,-3.25) -- (4.5,-3.25);
\draw (-6.5,-9.666667) -- (n16.west);
\draw (-6.5,-9.666667) -- (-6.5,-3.25);
\draw (n7.output) -- ++(right:15mm) node[midway, above, yshift=-2pt]{\scriptsize $\varphi_{0} = 0$};
\node[not gate] at (2.125,-4.5) (n46) {};
\draw (n46.output) -- (3.0,-4.5);
\node[circle, fill=black, inner sep=0pt, minimum size=3pt] (c0) at (1.0625,-4.5) {};
\draw (3,-4.5) -- (3,-5.5);
\draw (3,-5.5) -- (-6.75,-5.5);
\node[circle, fill=black, inner sep=0pt, minimum size=3pt] (c0) at (-6.75,-8.116667) {};
\draw (-6.75,-8.116667) -- (n20.west);
\draw (-6.75,-13.7) -- (n25.west);
\draw (-6.75,-13.7) -- (-6.75,-5.5);
\draw (1.0625,-4.5) -- (1.0625,-5.25);
\draw (1.0625,-5.25) -- (-7,-5.25);
\draw (-7,-6.333333) -- (n12.west);
\draw (-7,-6.333333) -- (-7,-5.25);
\draw (n10.output) -- ++(right:15mm) node[midway, above, yshift=-2pt]{\scriptsize $c_1 = 0$};
\draw (n13.output) -- ++(right:15mm) node[midway, above, yshift=-2pt]{\scriptsize $c_2 = 1$};
\draw (n21.output) -- ++(right:15mm) node[midway, above, yshift=-2pt]{\scriptsize $c_3 = 0$};
\draw (n34.output) -- ++(right:15mm) node[midway, above, yshift=-2pt]{\scriptsize $c_4 = 1$};
\end{tikzpicture}\end{center}
\begin{center}Цена схемы: $S_Q = 39$. Задержка схемы: $T = 5\tau$.\end{center}

\end{document}

